\documentclass[a4paper,12pt]{article}
\usepackage[utf8]{inputenc}
\usepackage{fullpage}
\usepackage[style=phys,maxbibnames=6]{biblatex}
\usepackage[skip=0.3em,indent=2em]{parskip}
\usepackage[pdftex]{graphicx}
\usepackage{float}
\usepackage{hyperref}
\usepackage{physics}
\usepackage{amsmath}
\usepackage{siunitx}
\usepackage{cleveref}


\addbibresource{references.bib}

\newcommand{\efn}{e4$\nu$}

\newcommand{\Mu}{$\mu$}
\newcommand{\Tau}{$\tau$}

\newcommand{\Ne}{$\nu_e$}
\newcommand{\Nm}{$\nu_\mu$}
\newcommand{\Nt}{$\nu_\tau$}

\newcommand{\Zz}{$Z^0$}
\newcommand{\Wp}{$W^+$}
\newcommand{\Wm}{$W^-$}
\newcommand{\pip}{$\pi^+$}
\newcommand{\pim}{$\pi^-$}


\begin{document}
\pagestyle{empty}
\par\noindent\includegraphics[width=12cm]{PandA_crest.pdf}
\par\noindent

\vspace*{2cm}

\begin{center}
        \Large\bf \Large\bf Senior Honours Project \\
        \LARGE\bf Further analysis of \efn\ data
\end{center}

\vspace*{0.5cm}

\begin{center}
        \bf Jan Kocka \\
        February 2022
\end{center}
\vspace*{5mm}

\begin{abstract}
    In this project we continue the work on data from \efn.
    We further inspect the data generated with GENIE, mainly looking at the events producing 1 pion and comparing then to data from the CLAS experiment.
\end{abstract}

\vspace*{1cm}

\subsubsection*{Declaration}

\begin{quotation}
  I declare that this project and report is my own work.
\end{quotation}

\vspace*{2cm}
Signature:\hspace*{8cm}Date:  31/10/2021

\vfill
{\bf Supervisor:} Dr. Cheryl Patrick
\hfill
10 Weeks
\newpage

\tableofcontents

\newpage

\pagestyle{plain}
\setcounter{page}{1}

\section{Background and Motivation}
This project concerns neutrinos and their properties, and so a brief description of the relevant properties and their scientific significance is presented.
Neutrinos are 3 uncharged leptons of different flavours (each corresponding to the other 3, charged, leptons -- electron, \Mu\ and \Tau\ of ascending masses).
Neutrinos interact only via the weak force and with very low cross sections, the Feynman diagrams of their interactions with the other fundamental particles through the W and Z bosons are shown in \cref{fig:nu_feyn}.

Their existence was originally proposed by Pauli in the 1930s to explain the beta decay, and their mass was long thought to be 0, most notably the standard model assumes this.
However at the end of the last century neutrinos have been observed to undergo oscillations which implies non-zero mass.
Since then they have been at the forefront of research, nowadays they are accepted to have mass and there are questions about them being their own antiparticles (Majorana) and possibly violating charge-parity symmetry.
The answers to these questions could significantly further our understanding of the universe and explain the dominance of matter over antimatter.

\begin{figure}[h]
    \centering
    \includegraphics{figures/NeutrinoFeynman.jpg}
    \caption{
        Feynman diagrams of weak force vertices for neutrino interactions, all other vertices including them can be obtain by suitable rotations of these
        (this requires replacing some particles with their antiparticles, including the \Wm becoming a \Wp).
        Say the first could be changed to represent a $e^-$ and \Wp\ colliding and forming a \Ne.
        Figures taken from \cite{potterFeynmanDiagramsParticlea}.
    }\label{fig:nu_feyn}
\end{figure}

\subsection{Neutrino Oscillations}
In particle physics oscillations is a phenomenon that takes place when the particle mass eigenstates which govern its evolution in time are not the same as the state in which it is observed.
This phenomenon is not exclusive to neutrinos, the neutral Kaon is known to oscillate to its own antiparticle and vice-versa \cite{burkhardtWavelengthNeutrinoNeutral2003}.
% However it might seem a bit abstract, but as we know from basic quantum mechanics when we make an observation, the observed system will collapse into some particular state depending on the result of our observation.

In the case of neutrinos, we say there are 3 flavours and the way we differentiate them is by the weak interaction, so say an electron capture takes place when an electron interacts with a proton to form a neutron and a neutrino.
This is a weak interaction process and so far all interactions we have observed conserve the 3 separate lepton numbers (electron, \Mu\ and \Tau\ again) thus we observe that neutrino to be an electron neutrino and we can say it is collapsed into the electron neutrino eigenstate.

Further, that electron eigenstate can be expressed in terms of the mass eigenstates, and if the 2 eigenstate sets aren't identical it will have multiple non-zero components.
Then as the mass eigenstates travel differently (the neutrino has a definite $E$ and so through $E^2 = m^2 + p^2$ we get that the mass eigenstates have different momenta and so speeds), the relative components of the state at different positions as it travels will change periodically -- oscillate.
Thus the probabilities of measuring the neutrino in any of the 3 flavour states also oscillates.
% The result is that if a neutrino is measured to be of some known flavour the probability of measuring it at a later time in any of the other flavour states changes and it might later be measured to be of a different flavour.

A very important property resulting from the two eigenstate sets being different is the transformation matrix between them.
For neutrinos this is the PMNS\footnotemark\ matrix, it is referred to as $U$ and is such that 
\begin{equation}
    \ket{\nu_\alpha} = \sum_i U_{\alpha i} \ket{\nu_i}, \text{ and }
    \ket{\nu_i} = \sum_\alpha U^\dag_{i \alpha} \ket{\nu_\alpha}
\end{equation}
where $\ket{\nu_\alpha}$ are the 3 flavour eigenstates and $\ket{\nu_i}$ the mass eigenstates.
The exact form of the matrix then determines many of the neutrino properties, obtaining accurate values for its elements is the main driving force of neutrino oscillation experiments.
For example if the matrix is not completely real then neutrinos violate CP symmetry.
\footnotetext{Stands for Pontecorvo–Maki–Nakagawa–Sakata matrix, it is the analogue of the CKM matrix for the mixing of quarks.}

Then to complete the picture, the probability of a neutrino measured to be of flavour $\alpha$ to be measured again in the flavour $\beta$ is given by 
\begin{equation} \label{eq:nosc}
    P(\alpha \rightarrow \beta) = \sum_i |U_{\alpha i} U^\dag_{i \beta}|^2 + 2\Re \sum_{j>i} U_{\alpha i} U^\dag_{i \beta} (U_{\alpha j} U^\dag_{j \beta})^* \exp(-i\frac{\Delta m^2_{ij}}{2}\frac{L}{E})
\end{equation}
where $E$ is it's energy, $L$ the distance between our measurements and $\Delta m^2_{ij}$ the mass difference of the mass eigenstates $i, j$ squared (for more detail see \cite{zuberNeutrinoPhysics2020}).
This is then our main entry point for studying the PMNS matrix from oscillation experiments, though it cannot actually be fully determined just from oscillation experiments (there may be another phase )

\subsection{Experiments and Energy Reconstruction}\label{sec:exanderec}
There are many active experiments (including Super-Kamiokande, OPERA, MINOS, DUNE and Hyper-Kamiokande\footnote{**maybe change which ones I mention and add references to them}) working on precise neutrino oscillation measurements to determine the neutrino masses and PMNS matrix elements.
In all of these experiments neutrinos from a source which has known fractions of the 3 flavours are beamed across large distances (for oscillations of 1\si{GeV} \Nm\ an $L$ of order $\sim 1000\si{km}$ is needed \cite{mezzettoThreeFlavorOscillationsAccelerator2020}) so that they oscillate and then some of the flavours (electron, \Mu\ or both) are observed and counted.
This means we are measuring some of the probabilities from \cref{eq:nosc} with a known $L$, however we don't have a simple value for $E$ which we need in order to use the data.
Experiments typically use either neutrinos from the sun, from the atmosphere due to the cosmic microwave background or from secondary neutrino beams at accelerators and all of these methods produce neutrinos with wide ranges of energies.
% We naturally don't have any control over the solar and atmospheric neutrinos, and while we can to some degree control the energies of neutrinos from accelerator sources not to the necessary degree.

Because of that experiments rely on reconstructing incident neutrino energies.
As neutrinos are uncharged and of small mass they aren't detected directly, instead a large amount (needed due to their small cross sections) of some stable substance (argon, water or chlorine) is stored and surrounded by detectors.
Then when neutrinos come in, some will interact with these atoms and produce detectable particles.
These always have to include a lepton of the corresponding flavour and possibly other particles, often pions or photons.
A collection of particle measurements of particles originating from one neutrino is called and event and these are grouped by different neutrino-nucleus interactions (the neutrinos don't scatter off of the atomic electrons).
This is done as they share many characteristics (for example what particles are produced), different interactions require different treatments for say the energy reconstructions and are fundamentally different on the theoretical level.

Which is where \efn\ comes in, in order to better interpret data from neutrino detectors and mainly to improve their energy reconstruction (which is becoming a significant factor on the uncertainties of the results) we need to understand these neutrino-nucleus interactions better.
However, as there are no monoenergetic neutrino beams it is very difficult to experimentally study these effects directly on neutrinos, so instead we study electron-nucleus interactions which are subject to the same quantum field theory interactions.

\subsubsection{Monte Carlo event generators}
**not done, but I think this might be a good place to introduce them


\subsection{Neutrino(Electron)-nucleus interactions}
First there is another concept to introduce, both neutrino and electron nucleus scattering can be described by
\begin{equation}
    l + A \rightarrow l' + a + b + c ...
\end{equation}
where an incoming lepton $l$ scatters off of a nucleus $A$ resulting in a lepton $l'$ and hadron particles $a, b, c ...$.
When considering realistic experiments we always know $l$ and $A$ but might only be able to detect the outgoing lepton and some or none of the outgoing hadrons. 
To differentiate these cases we consider separately inclusive cross sections where only $l'$ is detected (so the result is essentially as if integrating over all the possible outgoing hadrons).
Semi-inclusive cross sections where $l'$ and some of the hadrons are detected and exclusive cross sections where we know all of the resulting particles \cite{amaroElectronNeutrinonucleusScattering2020}.

As neutrinos only interact via the weak force, all neutrino-nucleus interactions take place via either the \Zz\ boson in which case the final lepton is the same neutrino, these are called Neutral Current (NC) and are not detected.
Or it can interact through the \Wp and \Wm\ bosons (Charged Current interactions) in which case a charged lepton is produced and detected.
This is contrasted to electrons which will dominantly interact with the nucleus through photons (so preserving the electron), nevertheless within quantum field theory both leptons interact via a vector current though neutrinos also interact through an axial-vector current \cite{alvarez-rusoNuSTEC11NeutrinoScattering2018}.
Due to this existing MC event generators like GENIE can be modified to generate electron-nucleus interaction events and test particularly the vector current component of the theories behind the generator.

Finally to come back to the interaction types mentioned in \cref{sec:exanderec}, the dominant types relevant for oscillation experiments are Quasi-elastic, Resonant and Deep Inelastic scaterring

\subsubsection{Quasi-elastic (QE) scattering}
These events are most dominant at lower energies, 


\subsubsection{Resonant scattering}
\subsubsection{Deep Inelastic scattering (DIS)}




\newpage


\section{Results \& Discussion}

\section{Conclusion}

\printbibliography

\end{document}
